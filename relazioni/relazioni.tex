\documentclass[a4paper,11pt]{report} 
\usepackage[utf8x]{inputenc}
\usepackage{amsmath} % AMS Math Package
\usepackage{amsthm} % Theorem Formatting
\usepackage{amssymb}	% Math symbols such as \mathbb
\usepackage{graphicx} % Allows for eps images
%\usepackage{multicol} % Allows for multiple columns
%\usepackage[dvips,letterpaper,margin=0.75in,bottom=0.5in]{geometry}
\usepackage{geometry}
\usepackage{booktabs}
\usepackage{enumerate}
\usepackage{verbatim}
\usepackage{fancyhdr}
\usepackage{titlesec}
\usepackage{longtable}
\usepackage{subfigure}
\usepackage{hyperref}
\pagestyle{fancy}
\fancyhf{}
%\fancyhead[LE,RO]{\slshape \rightmark}
\fancyhead[LO,RE]{\slshape \leftmark}
\fancyfoot[C]{\thepage}
%\titleformat{\chapter}%
%[display]{\vspace{2cm}\Large\bfseries}{\Huge\thechapter \ \hrulefill}{0pt}{\vspace{4mm}}%
%{\Large\bfseries\filcenter}

%%%%%%%%%%%%%%%%%%%%%%%%%%%%%%%%%%%%%%%%%%%%%%%%%%%%%%%%%%%%%%%%%%%%%%%%%%%%%%%%%%%%%%%%%%%%%%%%%%%%%%%%%%%%%%%%%%%%%%%%%%%%%%%%%%%%%%%%
						      %Mathematical Commands
%%%%%%%%%%%%%%%%%%%%%%%%%%%%%%%%%%%%%%%%%%%%%%%%%%%%%%%%%%%%%%%%%%%%%%%%%%%%%%%%%%%%%%%%%%%%%%%%%%%%%%%%%%%%%%%%%%%%%%%%%%%%%%%%%%%%%%%%%%
 % Sets margins and page size
%\pagestyle{empty} % Removes page numbers
\makeatletter % Need for anything that contains an @ command 
% \DeclareMathOperator{\Sample}{Sample}
\let\vaccent=\v % rename builtin command \v{} to \vaccent{}
\renewcommand{\v}[1]{\ensuremath{\mathbf{#1}}} % for vectors
\newcommand{\gv}[1]{\ensuremath{\mbox{\boldmath$ #1 $}}} 
% for vectors of Greek letters
\newcommand{\uv}[1]{\ensuremath{\mathbf{\hat{#1}}}} % for unit vector
\newcommand{\abs}[1]{\left| #1 \right|} % for absolute value
\newcommand{\avg}[1]{\left< #1 \right>} % for average
\let\underdot=\d % rename builtin command \d{} to \underdot{}
\renewcommand{\d}[2]{\frac{d #1}{d #2}} % for derivatives
\newcommand{\dd}[2]{\frac{d^2 #1}{d #2^2}} % for double derivatives
\newcommand{\pd}[2]{\frac{\partial #1}{\partial #2}} 
% for partial derivatives
\newcommand{\pdd}[2]{\frac{\partial^2 #1}{\partial #2^2}} 
% for double partial derivatives
\newcommand{\pdc}[3]{\left( \frac{\partial #1}{\partial #2}
 \right)_{#3}} % for thermodynamic partial derivatives
\newcommand{\ket}[1]{\left| #1 \right>} % for Dirac bras
\newcommand{\bra}[1]{\left< #1 \right|} % for Dirac kets
\newcommand{\braket}[2]{\left< #1 \vphantom{#2} \right|
 \left. #2 \vphantom{#1} \right>} % for Dirac brackets
\newcommand{\matrixel}[3]{\left< #1 \vphantom{#2#3} \right|
 #2 \left| #3 \vphantom{#1#2} \right>} % for Dirac matrix elements
\newcommand{\grad}[1]{\gv{\nabla} #1} % for gradient
\let\divsymb=\div % rename builtin command \div to \divsymb
\renewcommand{\div}[1]{\gv{\nabla} \cdot #1} % for divergence
\newcommand{\curl}[1]{\gv{\nabla} \times #1} % for curl
\let\baraccent=\= % rename builtin command \= to \baraccent
\renewcommand{\=}[1]{\stackrel{#1}{=}} % for putting numbers above =
\newtheorem{prop}{Proposition}
\newtheorem{thm}{Theorem}[section]
\newtheorem{lem}[thm]{Lemma}
\theoremstyle{definition}
\newtheorem{dfn}{Definition}
\theoremstyle{remark}
\newtheorem*{rmk}{Remark}
\usepackage[italian]{babel}
\usepackage{graphicx}
\usepackage{latexsym}
\usepackage{verbatim} % commenti estesi
\newcommand{\p}[2]{ \frac{ \partial #1}{ \partial #2}} %derivata partiale
\newcommand{\lap}{\nabla^2} %laplaciano
\newcommand{\ham}{\mathcal{H}} % Simbolo dell'hamiltoniana
\newcommand{\rint}{\int_\mathbb{R}} % Integrale su R
\newcommand{\modq}[1]{| #1|^2} % Modulo quadro di ``argomento''
\newcommand{\navg}[2]{\left< #1 ^{#2} \right>} %media di ``arg''^n
\newcommand{\con}[1]{\overline{#1}} %coniugato con la barra
\newcommand{\ih}{\frac{i}{\hbar}} % i/h tagliato
\newenvironment{sistem}%
{\left\lbrace\begin{array}{@{}l@{}}}%
{\end{array}\right.}



\usepackage{minted}
\title{Relazioni di Laboratorio di Fisica Computazionale}
\author{Carlo Sana}

\begin{document}

\maketitle
\section{Introduzione}


\section{Metodi di integrazione}
L'obiettivo di questa parte è di stimare il valore dell'integrale definito di una funzione di una variabile:
$$
 	I \ = \ \int_{x_{min}}^{x_{max}} f(x) \, dx
$$
Le prime routine che sono state scritte sono un'implementazione delle formule di Newton-Cotes (al primo e secondo ordine)
e sul metodo delle quadrature gaussiane.\\
Per aumentare la precisione del calcolo, il dominio di integrazione viene suddiviso in sottointervalli.
La larghezza di ogni intervallo è uniforme ed è possibile scegliere il numero di intervalli in cui
si vuole dividere il dominio di integrazione prima di chiamare le funzioni.
Per ottenere la stima dell'integrale è necessario sommare le stime degli integrali ottenute per i sottointervalli.\\
Nel nostro caso $n$ sarà il numero di sottointervalli. Definiamo così in modo naturale una partizione dell'insieme di integrazione:
$$
	h \ = \frac{x_{min} - x_{max}}{ n} \qquad \Longrightarrow \qquad a_i = x_{min} + i h  \qquad
	\mbox{ per } \ i = 0,1,....,n 
$$

La funzione che si occupa di questo compito è la seguente:
\begin{minted}{c}
double partition (double min, double max , int n, int methodNumber , double (*f) (double) ){
	double h  = 0;
	int i = 0;
	double Sum = 0;
	double (*method) (double ,double , double (*) (double));
	switch(methodNumber){
		case 1:
			method = trapezio;
			break;
		case 2 :
			method = Simpson;
			break;
		case 3 :
			method = gaussianQuad;
			break;
		default:
			printf("Bad integration method!");
			exit(EXIT_FAILURE);
			break;
	}
	h = (max-min)/n;
	for (i = 0; i< n; i++){
		Sum += method( min + i*h, min + (1+i)*h, f);
		}
	return Sum;
}
\end{minted}
E' necessario passare all'argomento della funzione il metodo di integrazione desiderato, attraverso un intero. 
Le funzioni che implementano i tre metodi di integrazione devono avere gli stessi parametri di input.
Nel caso gli estremi di integrazione sono scambiati, ossia $x_min > x_max$, l'integrazione avviene comunque correttamente, visto che in questo caso $h$ sarà negativo.
\subsection*{Newton-Cotes}
Le formule di Newton-Cotes si ottengono interpolando la funzione integranda con polinomi di Lagrange.
Il polinomio di Lagrange j-esimo di grado n è definito come:
$$
	l_j^n \ = \ \prod_{i = 0, i \ne j}^{n} \frac{x-x_i}{x_j - x_i}
$$
	Come si può vedere è un polinomio di grado n con la proprietà:
$$
	l_j^n ( x_j) = \delta_{ij}
$$
E' ora immediato costruire un polinomio $P(x)$ tale che $ P(x_i) = f(x_i) \qquad \forall  0<i<n$.
Questo polinomio è il seguente:
$$
	P(x) = \sum_{i = 0}^n f(x_i) l_i^n(x)
$$
La stima dell'integrale diventa così:
$$
 I \ = \ \int_{x_{min}}^{x_{max}} P(x) \, dx \ \ = \ \int_{x_{min}}^{x_{max}} \sum_{i = 0}^n f(x_i) l_i^n(x)  \ =  \ \sum_{i = 0}^n f(x_i) \int_{x_{min}}^{x_{max}} l_i^n(x) \,dx 
$$
 Si dimostra inoltre che, con un cambio di variabile:
 $$
 \omega_j = \int_{x_{min}}^{x_{max}} l_i^n(x) \,dx  = \int_{0}^{n} \prod_{i = 0 \ i \ne j}^n \frac{z-i}{j-i}\,dz 
$$
Questa è ovviamente una stima dell'integrale e si dimostra che l'errore, utilizzando $n+1$ punti è uguale a :
$$
 E_n = \frac{1}{(n+1)!} \int_{x_{min}}^{x_{max}} f^{n+1}(\xi) \prod_{i = 0}^n \frac{x-x_i} \, dx
$$
dove $\xi$ è un punto interno all'intervallo. L'errore è facilmente sovrastimabile, valutando il massimo della derivata $n+1$-esima all'interno dell'intervallo.
\subsubsection*{Newton-Cotes:1°ordine}
Questo metodo consiste nell'approssimare la funzione fra due punti $a_i$ e $a_{i+1}$ con un segmento.
L'area si ottiene calcolando l'area del trapezio sotteso da questo segmento, oppure applicando le formule di Newton-Cotes, ponendo $ n = 1$: 
$$
\omega_0 \ = \ \frac{1}{2} \qquad \omega_1 \ = \ \frac{1}{2}
$$
La routine che implementa la formula è la seguente:
\begin{minted}{c}
double trapezio ( double min , double max, double (*f) (double) ){
	return ((max-min)*( f(min) + f(max) )/2.0 ) ;
	}
\end{minted}
dove $f$ è il puntatore a funzione della funzione integranda.

\subsection*{Newton-Cotes:2°ordine}
In questo caso, l'approssimazione viene fatta con polinomi di grado 2, ossia parabole.
I "pesi" $\omega_i$ valgono:
$$
	\omega_0 = \frac{1}{6} \qquad  \omega_1 = \frac{2}{3} \qquad \omega_2 = \frac{1}{6}
$$
\begin{minted}{c}
double Simpson ( double min , double max, double (*f) (double) ){
		return ((max-min)*( f(min) + 4.0*f(min + (max-min)/2) + f(max))/6.0);
	}
\end{minted}

\subsection*{Quadrature gaussiane}
Nel caso delle quadrature gaussiane, i punti della partizione non vengono più scelti
"cieca" in modo da essere equidistanti, ma vengono scelti in maniera più opportuna.\\
E' fondamentale l'uso di polinomi ortogonali in un certo intervallo $[a,b]$ con il peso $\omega(x)$:
$$
	\int_a^b \omega(x)\ P_n{x} \ P_m{x} \ = \ \delta_{m,n}
$$
Inoltre, un teorema afferma che un polinomio ortogonale di grado $n$ in $[a,b]$ ha $n$ zeri in $[a,b]$.\\

\begin{minted}{c}
double gaussianQuad ( double min , double max , double (*f) (double)){
	/*Viene usato un polinomio di grado 5 */
	double zero[5] = {0 ,
			sqrt(245.0 - 14.0*sqrt(70.0))/21.0,
			-sqrt(245.0 - 14.0*sqrt(70.0))/21.0,
			sqrt(245.0 + 14.0*sqrt(70.0))/21.0,
			-sqrt(245.0 + 14.0*sqrt(70.0))/21.0};
	double weight[5] = {(double)128.0/225.0,
			(double)1/900.0*(322.0 + 13*sqrt(70.0)),
			(double)1/900.0*(322.0 + 13*sqrt(70.0)),
			(double)1/900.0*(322.0 - 13*sqrt(70.0)),
			(double)1/900.0*(322.0 - 13*sqrt(70.0))} ;
	double integral =  0;
	/* Porto [min,max] in [-1,1] */
	int i = 0;
	for (i = 0 ; i< 5 ; i++){
		integral += weight[i]*f((max-min)/2.0*zero[i]+(max+min)/2.0);
	}
	return integral*(max-min)/2.0 ;
}
\end{minted}
\section{Integrali di cammino nell'oscillatore armonico}



\end{document}